%%% -*- TeX-engine: xetex; Tex-PDF-mode: t -*-

\documentclass[11pt,a4paper,sans]{moderncv}

% adjust the page margins
\usepackage[scale=0.75]{geometry}
\moderncvstyle{classic}
\moderncvcolor{blue}
%\setlength{\hintscolumnwidth}{2cm}

\definecolor{UrlBlue}{RGB}{56, 115, 179}
\definecolor{MinusBlue}{RGB}{149, 180, 214}
\newcommand{\urlblue}[1]{\textcolor{UrlBlue}{#1}}
\newcommand{\minusblue}[1]{\textcolor{MinusBlue}{#1}}
\usepackage{fontspec}
\usepackage{xunicode}
\usepackage{xeCJK}
%\xeCJKsetup{AutoFakeBold=true, AutoFakeSlant=true, EmboldenFactor=1}
\xeCJKsetup{AutoFakeSlant=true}
\setCJKmainfont{WenQuanYi Micro Hei}
\setCJKmonofont{WenQuanYi Micro Hei Mono}

% adjust the page margins
\usepackage[scale=0.75]{geometry}

% personal data
\name{王文鑫}{}
\title{简历}
\address{清华大学}{电子工程系}{}
\phone[mobile]{188~1136~9901}
\email{wenxin.wang94@gmail.com}
\social[github]{wenxin-wang}
\photo[64pt][0.4pt]{figs/photo.jpg}

\begin{document}
\makecvtitle

\section{课程}
\cvlistdoubleitem{计算机程序设计基础 (C++)}{数据与算法 (C++)}
\cvlistdoubleitem{计算机网络}{操作系统}
\cvlistdoubleitem{数据库}{}

\section{经历}
\subsection{项目}

\cventry{2016.03--}{IPv4/6翻译技术的流量调度}{李星}{清华大学网络中心}{}{
  升级IPv4/6翻译器(RFC6219)。该项目及后续工作是我研究生阶段的主要课题。\newline{}
  目的:
  \begin{itemize}
  \item 允许翻译器细粒度地选择各个流的翻译模式和上行出口
  \item 对各翻译器的选择策略进行中心控制
  \end{itemize}
  使用Netfilter和路由实现策略选择和翻译\newline{}
  语言:C,Shell,Python
}

\vspace{0.3cm}
%\cventry{2016.01--2016.02}{使用应用层代理解决老旧网站的IPv6访问}{李星}{清华大学网络中心}{}{}

\cventry{2015.10--2015.12}{Openstack IPv6环境搭建与测试}{李星}{清华大学网络中心}{}{
  在Openstack Liberty版本中完成了Slaac, DHCPv6 Stateless和DHCPv6Stateful环境的配
  置和测试。(\href{https://github.com/wenxin-wang/puppet-phystack}{\urlblue{Github}})\newline{}
  上游接受的补丁:
  \begin{itemize}
  \item \href{https://review.openstack.org/262976}{add ipv6 options to neutron\_subnet type}
  \item \href{https://review.openstack.org/261418}{provider/neutron.rb: fix list\_router\_ports}
  \item \href{https://review.openstack.org/262962}{radvd prefix configuration for DHCPV6 Stateful RA}
  \end{itemize}
  语言:Puppet,Python,Ruby
}

\vspace{0.3cm}
\cventry{2015.08--2015.10}{新版CERNET2网管系统开发}{张辉}{清华大学网络中心}{}{
  结合目前网管系统缺陷,为教育网提供新的网管系统。系统包含设备管
  理、网络气象图的编辑和呈现等功能。\newline{}
  贡献代码7000行左右,试运行中。\newline{}
  语言:Python,Javacript
}

\vspace{0.5cm}
\subsection{科研}
\cventry{2015.04--2015.06}{NFV的一致性更新}{李勇}{清华大学电子系}{}{
  以Floodlight作为平台,研究SDN环境中NFV(Middlebox)的快速一致迁移问题。提出了基本思路,改进后的算法相比
  现有算法在性能和稳定性方面有很大提升。\newline{}
  作为论文二作,负责算法实现。论文投往CoNEXT2015,算法的效率得到评委肯定。\newline{}
  语言:Java,Shell
}

\vspace{0.3cm}
\cventry{2014.10--2015.01}{利用Docker和OpenBTS搭建虚拟GSM基站池}{}{清华大学电子系}{}{
  利用Linux容器搭建可动态分配资源的轻量化虚拟GSM基站池。
  最终实现两个基站同时运行于一台宿主机,以及核心网与基站分离的原型。\newline{}
  语言:Shell,C++
}

\section{编程语言}
\cvitem{熟悉}{C++, Shell, Python, Javascript}
\cvitem{了解}{Java}

\section{英语}
\cvitem{}{英语能力足以胜任日常的工作和科研需求}
\cvitem{CET-6}{644}
\cvitem{TOFEL}{110}
\cvitem{TEPT}{84}

\section{相关能力}
\cvitem{}{有维护Linux服务器的经验,熟悉网络和虚拟化配置}
\cvitem{}{
有较多的项目开发经验,熟悉工程实践,具备与人交流协调的能力。
项目内容跨度较广,可以快速学习新的知识并投入应用}
\cvitem{}{了解Linux操作系统平台,熟悉开发和编译工具链,了解Linux网络编程}

\section{时间}
\cvitem{每周时间}{除去毕设答辩的时间节点(4月12日左右中期答辩,6月中旬结题答
  辩)可能需要2-3天准备时间,每周可以工作4-5天}
\cvitem{实习时长}{4月可以开始工作,至少工作至7月,如果时间允许可以继续实习}

\end{document}