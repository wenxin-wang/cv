%%% -*- TeX-engine: xetex; Tex-PDF-mode: t -*-

\documentclass[11pt, a4paper, sans]{moderncv}

% adjust the page margins
\usepackage[scale=0.75]{geometry}
\moderncvstyle{classic}
\moderncvcolor{blue}
%\setlength{\hintscolumnwidth}{2cm}

\definecolor{UrlBlue}{RGB}{56, 115, 179}
\definecolor{MinusBlue}{RGB}{149, 180, 214}
\newcommand{\urlblue}[1]{\textcolor{UrlBlue}{#1}}
\newcommand{\minusblue}[1]{\textcolor{MinusBlue}{#1}}
\usepackage{fontspec}
\usepackage{xunicode}
\usepackage{xeCJK}
%\xeCJKsetup{AutoFakeBold=true, AutoFakeSlant=true, EmboldenFactor=1}
\xeCJKsetup{AutoFakeSlant=true}
\setCJKmainfont{WenQuanYi Micro Hei}
\setCJKmonofont{WenQuanYi Micro Hei Mono}

% personal data
\name{王文鑫}{}
\title{简历}
\address{清华大学}{电子工程系}{}
\phone[mobile]{188~1136~9901}
\email{wenxin.wang94@gmail.com}
\social[github]{wenxin-wang}
% \photo[64pt][0.4pt]{figs/photo.jpg}

\begin{document}
\makecvtitle

\section{教育}
\cventry{2012.09--2016.07}{\mdseries 本科(在读)}{清华大学}{}{}{
  清华大学电子工程系\, 电子信息科学与技术专业
}
\cvitem{推免}{清华大学网络研究院\, 信息与通信工程专业\, 硕士研究生}

\cvitem{课程}{\minusblue{--} 计算机程序设计基础 (C++)}
\cvlistdoubleitem{数据与算法 (C++)}{计算机网络}
\cvlistdoubleitem{操作系统}{数据库}
\cvitem{TOFEL}{110}

\section{经历}
\subsection{项目}

\cventry{2016.03--现在}{\mdseries IPv4/6翻译技术的流量调度}{}{清华大学网络研究院}{}{
  升级IPv4/6翻译器(RFC6219)
  \begin{itemize}
  \item 允许翻译器细粒度地选择各个流的翻译模式和上行出口
  \item 对各翻译器的选择策略进行中心控制
  \end{itemize}
  使用Netfilter和路由实现策略选择和翻译\\
  语言:C,Shell,Python
}

\vspace{0.3cm}
%\cventry{2016.01--2016.02}{使用应用层代理解决老旧网站的IPv6访问}{}{清华大学网络研究院}{}{}

\cventry{2015.10--2015.12}{\mdseries Openstack IPv6环境搭建与测试}{}{清华大学网络研究院}{}{
  在Openstack Liberty版本中实现了Slaac、DHCPv6 Stateless、DHCPv6Stateful\\
  三种IPv6环境的配置和测试(\href{https://github.com/wenxin-wang/puppet-phystack}{\urlblue{Github}})。\\
  个人提交并被官方合并的补丁:
  \href{https://review.openstack.org/262976}{\urlblue{262976}},
  \href{https://review.openstack.org/261418}{\urlblue{261418}},
  \href{https://review.openstack.org/262962}{\urlblue{262962}}\\
  语言:Puppet,Python,Ruby
}

\vspace{0.3cm}
\cventry{2015.08--2015.10}{\mdseries 新版CERNET2网管系统开发}{}{清华大学网络研究院}{}{
  为教育网提供易于使用的新版网管系统,贡献代码7000行左右,试运行中。\\
  语言:Python,Javacript
}

\vspace{0.3cm}
\cventry{2013.08--2013.10}{\mdseries 2013年第15届清华大学电子设计竞赛官方网站制作}{}{}{}{
  独立实现前端和后端。网站实现了队伍和赛事数据处理、新闻公告、文件传输和论坛功能,为比赛主办方继续沿用。\\
  语言:Ruby
}

\newpage{}
\subsection{科研}
\cventry{2015.04--2015.06}{\mdseries NFV的一致性更新}{}{清华大学未来网络与通信实验室}{}{
  研究SDN环境中NFV的快速一致迁移问题。\\
  提出了基本思路,改进后的算法相比现有算法在性能和稳定性方面有很大提升。作为论文二作,负责算法实现。\\
  Yujie Liu, \textbf{Wenxin Wang}, Jiaqiang Liu, Yong Li, Haoyu Song, Yue Wang, and Jian
  Yuan, \textit{Consistent State Updates for Virtualized Network Function Migration},
  IEEE Transactions on Services Computing (under review).\\
  语言:Java,C,Shell
}

\vspace{0.3cm}
\cventry{2014.10--2015.01}{\mdseries 利用Docker和OpenBTS搭建虚拟GSM基站池}{}{网络融合实验室}{}{
  利用Linux容器搭建可动态分配资源的轻量化虚拟GSM基站池。\\
  最终实现两个基站同时运行于一台宿主机,以及核心网与基站分离的原型。\\
  语言:Shell,C++
}

\vspace{0.3cm}
\cventry{2014.04--2014.06}{\mdseries 虚拟机嵌套文件系统性能分析}{}{清华大学计算机系}{}{
  研究由于宿主机和虚拟机嵌套文件系统导致的读写性能下降问题。\\
  抓取各层的IO操作,进行各层事件匹配,找到从虚拟机到宿主机的操作映射模式。\\
  语言:Shell,Python
}

\section{编程语言}
\cvitem{精通}{Shell,C++,Python,Javascript}
\cvitem{熟悉}{Java,Ruby}

\section{相关能力}
\cvitem{Linux}{有4年(deb、rpm系)Linux环境的工作经验}
\cvitem{}{熟悉开发和编译工具链,完成(Cross)Linux From Scratch编译和运行}
\cvitem{}{了解Linux网络编程}
\vspace{0.3cm}
\cvitem{运维能力}{熟悉网络和虚拟化配置,使用libvirt运行日常工作和测试环境}
\cvitem{}{熟悉自动部署,熟悉Ansible和Puppet使用}
\vspace{0.3cm}
\cvitem{开发能力}{有较多的项目开发经验,熟悉工程实践,具备与人交流协调的能力}
\cvitem{}{项目内容跨度较广,可以快速学习新的知识并投入应用}

% \section{时间}
% \cvitem{每周时间}{除去毕设答辩的时间节点(4月12日左右中期答辩,6月中旬结题答
%   辩)可能需要2-3天准备时间,每周可以工作4-5天}
% \cvitem{实习时长}{4月可以开始工作,至少工作至7月,如果时间允许可以继续实习}

\end{document}