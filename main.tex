%%% -*- TeX-engine: xetex; Tex-PDF-mode: t -*-

\documentclass[11pt, a4paper, sans]{moderncv}

% adjust the page margins
\usepackage[scale=0.75]{geometry}
\moderncvstyle{classic}
\moderncvcolor{blue}
%\setlength{\hintscolumnwidth}{2cm}

\definecolor{UrlBlue}{RGB}{56, 115, 179}
\definecolor{MinusBlue}{RGB}{149, 180, 214}
\newcommand{\urlblue}[1]{\textcolor{UrlBlue}{#1}}
\newcommand{\minusblue}[1]{\textcolor{MinusBlue}{#1}}
\usepackage{fontspec}
\usepackage{xunicode}
\usepackage{xeCJK}
%\xeCJKsetup{AutoFakeBold=true, AutoFakeSlant=true, EmboldenFactor=1}
\xeCJKsetup{AutoFakeSlant=true}
\setCJKmainfont{WenQuanYi Micro Hei}
\setCJKmonofont{WenQuanYi Micro Hei Mono}

\usepackage{xpatch}% http://ctan.org/pkg/xpatch
% \xpatchcmd{<cmd>}{<search>}{<replace>}{<success>}{<failure>}
\xpatchcmd{\cvitem}{\raggedleft\hintstyle{#2}}{\raggedright\hintstyle{#2}}{}{}

% personal data
\name{王文鑫}{}
\address{快手}{音视频传输算法组}{}
\phone[mobile]{188~1136~9901}
\email{i@wenxinwang.me}
\social[github]{wenxin-wang}
% \photo[64pt][0.4pt]{figs/photo.jpg}

\begin{document}
\makecvtitle

\vspace*{-1cm}% Insert needed vertical retraction

\section{教育经历}
\cventry{2012.09--2016.07}{\mdseries 电子工程系学士}{清华大学}{}{电子信息科学与技术专业}{}
\cventry{2016.09--2019.07}{\mdseries 网络研究院硕士}{清华大学}{}{信息与通信工程专业}{}

\cvitem{课程}{计算机程序设计基础 (C++),数据与算法 (C++)}
\cvitem{}{计算机网络,操作系统,数据库,模式识别}
\cvitem{TOFEL}{110}

\section{工作经历}

\cventry{2020.11--2021.04}{\mdseries 海外拉流体验优化}{快手}{}{}{
  \begin{itemize}
  \item 设计实现基于私有拉流协议的服务器/客户端 SDK,结合业务特点进行丢帧策略改进、连接复用等优化
  \item 32 核服务器,负载 50\% 时可以承载万兆流量
  \item 开播失败率降低 10\%,卡顿率降低 20\%
  \end{itemize}
  语言:C++,Shell
}

\cventry{2019.10--2020.11}{\mdseries 基于 P2P 的 RTC 体验优化}{快手}{}{}{
  \begin{itemize}
  \item NAT 穿透,P2P RTC 穿透率 60\%,P2P 直播 80\%
  \item P2P RTC 研发,成本节约 44\%,延迟降低 2\%
  \end{itemize}
  语言:C++,Shell,Java
}

\section{项目经历}

\cventry{2018.12--2019.02}{\mdseries IPv4/IPv6 翻译系统(RFC6219)研发}{清华大学网络研究院}{}{}{
  \begin{itemize}
  \item 内核态的数据包翻译模块,基于 Linux 虚拟设备和 netfilter,支持 x86/arm
  \item 使用 exabgp 开发基于 BGP 协议的翻译器控制信令
  \end{itemize}
  语言:C,Shell,Python
}

\cventry{2018.11--2019.01}{\mdseries 基于 Flowlabel 的 IPv6 流量工程}{清华大学网络研究院}{}{}{
  \begin{itemize}
  \item 利用 IPv6 的 Flowlabel 域为数据包增加标记,在网络中依据此标记进行转发
  \item 使用 Linux eBPF(xdp 和 lwt tunnel)处理和转发数据帧
  \end{itemize}
  语言:C,Shell, eBPF
}

\cventry{2018.10--2018.12}{\mdseries IPv4/IPv6 翻译场景下用户接入网的无感知认证系统}{清华大学网络研究院}{}{}{
  \begin{itemize}
  \item 利用校园网邮箱和设备 MAC 地址,在入网时进行设备认证
  \item 利用 Linux conntrack 模块进行请求溯源
  \item 基于 Systemd 和 Docker 的服务管理
  \end{itemize}
  语言:C,Shell,Ansible
}

\cventry{2018.10--2015.12}{\mdseries 树莓派网络探针系统}{清华大学网络研究院}{}{}{
  \begin{itemize}
  \item 部署在全国各高校 CERNET 节点,采集到各个运营商的网络情况
  \item 自动化批量生产,使用只读文件系统延长节点寿命,可远程控制和更新
  \end{itemize}
  语言:Shell,Python,Saltstack
}

% \cventry{2017.07--2018.05}{\mdseries 可靠文件传输协议开发}{快手}{}{}{
%   设计与实现弱网环境下的大吞吐量文件传输协议,用于短视频上传,在线
%   上系统进行灰度测试\\
%   语言:C++,Java,Shell }

% \cventry{2016.04--2016.07}{\mdseries 高性能分布式存储系统调研}{京东}{}{}{
%   调研和整理Apache Kudu的存储实现细节,以及Alluxio、Apache Spark等基本的工作流程
%   、储存读取方式和作为缓存的使用方法\\
%   语言:C++,Java
% }

% \cventry{2016.03--2016.06}{\mdseries IPv4/6翻译技术的流量调度}{清华大学网络研究院}{}{}{
%   升级IPv4/6翻译器(RFC6219),基于Linux内核
%   \begin{itemize}
%   \item 允许翻译器细粒度地选择各个流的翻译模式和上行出口
%   \item 对各翻译器的选择策略进行中心控制
%   \end{itemize}
%   % 使用Netfilter和路由实现策略选择和翻译\\
%   语言:C,Shell,Python
% }

%\cventry{2016.01--2016.02}{使用应用层代理解决老旧网站的IPv6访问}{}{清华大学网络研究院}{}{}

\cventry{2015.10--2015.12}{\mdseries Openstack IPv6环境搭建与测试}{清华大学网络研究院}{}{}{
  在Openstack Liberty版本中实现了Slaac、DHCPv6 Stateless、DHCPv6Stateful\\
  三种IPv6环境的配置和测试(\href{https://github.com/wenxin-wang/puppet-phystack}{\urlblue{Github}})。\\
  被官方合并的补丁:
  \href{https://review.openstack.org/262976}{\urlblue{262976}},
  \href{https://review.openstack.org/261418}{\urlblue{261418}},
  \href{https://review.openstack.org/262962}{\urlblue{262962}}\\
  语言:Puppet,Python,Ruby
}

% \cventry{2015.08--2015.10}{\mdseries 新版CERNET2网管系统开发}{清华大学网络研究院}{}{}{
%   为教育网提供易于使用的新版网管系统,贡献代码7000行左右,试运行中。\\
%   语言:Python,Javacript
% }

% \cventry{2013.08--2013.10}{\mdseries 第15届清华大学电子设计大赛官方网站制作}{清华大学电子系科协}{}{}{
%   独立完成网站的制作,功能包含队伍和赛事数据处理、新闻公告、文件传输和论坛,
%   代码在后续赛事中沿用。\\
%   语言:Ruby
% }

\section{科研经历}
\cventry{2015.04--2017.10}{\mdseries NFV的一致性更新(一作)}{清华大学未来网络与通信实验室}{}{}{
  研究SDN环境中NFV的快速一致迁移问题。改进后的算法相比现有算法在性能提升 3 倍。\\
  W. Wang, Y. Liu, Y. Li, H. Song, Y. Wang and J. Yuan, \textit{Consistent State
    Updates for Virtualized Network Function Migration}, in IEEE Transactions on
  Services Computing, vol. PP, no. 99, pp. 1-1. \\
  语言:Java,C,Shell
}

%\vspace{0.3cm}
%\cventry{2014.10--2015.01}{\mdseries 虚拟GSM基站池}{清华
%  大学网络融合实验室}{}{}{
%  利用Docker和OpenBTS搭建可动态分配资源的轻量化虚拟GSM基站池。\\
%  最终实现两个基站同时运行于一台宿主机,以及核心网与基站分离的原型。\\
%  语言:Shell,C++
%}

%\vspace{0.3cm}
%\cventry{2014.04--2014.06}{\mdseries 虚拟机嵌套文件系统性能分析}{清华大学计算机系}{}{}{
%  研究由于宿主机和虚拟机嵌套文件系统导致的读写性能下降问题。\\
%  记录各层的IO操作,进行事件匹配,找到从虚拟机到宿主机的操作映射模式。\\
%  语言:Shell,Python
%}

% \section{相关能力}
% \cvitem{Linux}{7年(deb、rpm系)Linux环境的工作经验,熟悉开发和编译工具链}
% \cvitem{运维能力}{熟悉网络和虚拟化配置,使用Libvirt和LXC运行日常工作和测试环境}
% \cvitem{}{熟悉自动部署,熟悉Ansible和Saltstack使用}
% \cvitem{开发能力}{有较多的项目开发经验,有良好的沟通协调能力及团队合作精神}
% \cvitem{}{项目内容跨度较广,可以快速学习新的知识并投入应用}

% \section{时间}
% \cvitem{每周时间}{除去毕设答辩的时间节点(4月12日左右中期答辩,6月中旬结题答
%   辩)可能需要2-3天准备时间,每周可以工作4-5天}
% \cvitem{实习时长}{4月可以开始工作,至少工作至7月,如果时间允许可以继续实习}

\end{document}
