%%% -*- TeX-engine: xetex; Tex-PDF-mode: t -*-

\documentclass[11pt, a4paper, sans]{moderncv}

% adjust the page margins
\usepackage[scale=0.75,tmargin=1cm,bmargin=1cm,footnotesep=1cm]{geometry}
\moderncvstyle{classic}
\moderncvcolor{blue}
%\setlength{\hintscolumnwidth}{2cm}

\definecolor{UrlBlue}{RGB}{51, 153, 255}
\newcommand{\urlblue}[1]{\textcolor{UrlBlue}{#1}}
\definecolor{MinusBlue}{RGB}{149, 180, 214}
\newcommand{\minusblue}[1]{\textcolor{MinusBlue}{#1}}
\usepackage{fontspec}
\usepackage{xunicode}
\usepackage{xeCJK}
%\xeCJKsetup{AutoFakeBold=true, AutoFakeSlant=true, EmboldenFactor=1}
\xeCJKsetup{AutoFakeSlant=true}
\setCJKmainfont{WenQuanYi Micro Hei}
\setCJKmonofont{WenQuanYi Micro Hei Mono}

\usepackage{xpatch}% http://ctan.org/pkg/xpatch
% \xpatchcmd{<cmd>}{<search>}{<replace>}{<success>}{<failure>}
\xpatchcmd{\cvitem}{\raggedleft\hintstyle{#2}}{\raggedright\hintstyle{#2}}{}{}

% personal data
\name{Wenxin Wang}{}
\title{Resume}
\address{Tsinghua University}{Department of Electronics Engineering}{}
\phone[mobile]{188~1136~9901}
\email{i@wenxinwang.me}
\social[github]{wenxin-wang}
% \photo[64pt][0.4pt]{figs/photo.jpg}

\begin{document}
\makecvtitle

\section{Education}
\cventry{2012.09--2016.07}{\mdseries Undergraduate, Department of Electronics
  Engineering}{Tsinghua University}{}{}{}
\cventry{2016.09--}{\mdseries Graduate, School of Networking}{Tsinghua Univerisity}{}{}{}

\cvitem{Courses}{Computer Program Design (C++), Data Structure and Algorithms (C++)}
\cvitem{}{Computer Network, Operating System, Database, Pattern Recognition}
\cvitem{TOFEL}{110}

\section{Experiences}
\subsection{Internship and Projects}

\cventry{2017.07--2018.05}{\mdseries Reliable Upload Protocol}{Kuaishou}{}{}{
  Design and implementation of a video file upload protocol that works reliably under
  significant packet loss and delay, evaluated in A/B tests. \\
  Programming Languages: C++, Java, Shell
}

\cventry{2016.04--2016.07}{\mdseries Study of the Implementation of Apache Kudu's
Storage System}{Jingdong}{}{}{
  Study the details of the storage system of Apache Kudu, along with cache
  systems such as Alluxio and Apache Spark, and their potential usage together with
  Apache Kudu\\
  Programming Languages: C++, Java
}

\cventry{2016.03--2016.06}{\mdseries Traffic Engineering with IPv4/6
  translation}{School of Networking, Tsinghua}{}{}{
  Upgrade our kernel-based IPv4/6 translator(RFC6219), to support
  \begin{itemize}
  \item Fine-grained selection of the translation scheme and link for each flow
  \item Centralized control of each translator's traffic engineering function
  \end{itemize}
  % 使用Netfilter和路由实现策略选择和翻译\\
  Programming Languages: C, Shell, Python
}

%\cventry{2016.01--2016.02}{使用应用层代理解决老旧网站的IPv6访问}{}{清华大学网络研究院}{}{}

\cventry{2015.10--2015.12}{\mdseries Evaluation of Openstack's IPv6
  Functions}{School of Networking, Tsinghua}{}{}{
  Evaluation of common schemes of IPv6 address configuration (Slaac, DHCPv6
  Stateless, DHCPv6 Stateful) in Openstack Liberty\\
  (\href{https://github.com/wenxin-wang/puppet-phystack}{\urlblue{Github}})。\\
  Patches contributed to upstream:
  \href{https://review.openstack.org/262976}{\urlblue{262976}}, 
  \href{https://review.openstack.org/261418}{\urlblue{261418}}, 
  \href{https://review.openstack.org/262962}{\urlblue{262962}}\\
  Programming Languages: Puppet, Python, Ruby
}

% \cventry{2015.08--2015.10}{\mdseries 新版CERNET2网管系统开发}{清华大学网络研究院}{}{}{
%   为教育网提供易于使用的新版网管系统, 贡献代码7000行左右, 试运行中。\\
%   语言:Python, Javacript
% }

% \cventry{2013.08--2013.10}{\mdseries 第15届清华大学电子设计大赛官方网站制作}{清华大学电子系科协}{}{}{
%   独立完成网站的制作, 功能包含队伍和赛事数据处理、新闻公告、文件传输和论坛, 
%   代码在后续赛事中沿用。\\
%   语言:Ruby
% }

\subsection{Research}
\cventry{2015.04--2017.10}{\mdseries Consistent Migration of NFV}{Future
  Internet Lab, Tsinghua}{}{}{
  Design and Evaluation of a fast and stable algorithm for consitent NFV migration.\\
  W. Wang, Y. Liu, Y. Li, H. Song, Y. Wang and J. Yuan, \textit{Consistent State
    Updates for Virtualized Network Function Migration}, in IEEE Transactions on
  Services Computing, vol. PP, no. 99, pp. 1-1. \\
  Programming Languages: Java, C, Shell
}

%\vspace{0.3cm}
%\cventry{2014.10--2015.01}{\mdseries 虚拟GSM基站池}{清华
%  大学网络融合实验室}{}{}{
%  利用Docker和OpenBTS搭建可动态分配资源的轻量化虚拟GSM基站池。\\
%  最终实现两个基站同时运行于一台宿主机, 以及核心网与基站分离的原型。\\
%  语言:Shell, C++
%}

%\vspace{0.3cm}
%\cventry{2014.04--2014.06}{\mdseries 虚拟机嵌套文件系统性能分析}{清华大学计算机系}{}{}{
%  研究由于宿主机和虚拟机嵌套文件系统导致的读写性能下降问题。\\
%  记录各层的IO操作, 进行事件匹配, 找到从虚拟机到宿主机的操作映射模式。\\
%  语言:Shell, Python
%}

\section{Programming Language}
\cvitem{Daily Usage}{Shell, C/C++, Python, Javascript}

\section{Relevant Skills}
\cvitem{Linux}{7-year working experience with Linux and compilation toolchains}
\cvitem{DevOps}{Daily usage of networking and virtualization tools
  (Libvirt, LXC, Docker)}
\cvitem{}{Ansible and Saltstack for testing and production}
\cvitem{Engineering}{Communication skills and teamwork experience, Always ready to learn}

% \section{时间}
% \cvitem{每周时间}{除去毕设答辩的时间节点(4月12日左右中期答辩, 6月中旬结题答
%   辩)可能需要2-3天准备时间, 每周可以工作4-5天}
% \cvitem{实习时长}{4月可以开始工作, 至少工作至7月, 如果时间允许可以继续实习}

\end{document}